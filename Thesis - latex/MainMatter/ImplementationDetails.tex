\chapter{Detalles de Implementación}
  Para la implementación del sistema propuesto en el capítulo anterior, se
decidió utilizar \emph{Python}\footnote{\url{http://www.python.org}} como
lenguaje de programación. Esta decisión permite aprovechar las facilidades para
el trabajo con expresiones regulares y la integración con los sistemas
\emph{Linux} que brindan la librería estándar del lenguaje. La programación del
sistema en \emph{Python} permite el uso del módulo \texttt{pyro} para realizar
la comunicación entre los agentes mediante un modelo \emph{RMI}.

La implementación del sistema fue concebida mediante el desarrollo de clases que
representaran las diferentes componentes de diseño que lo integran. Estas clases
representan el comportamiento de cada componente en el sistema
independientemente del proceso de análisis relativo a cada servicio. Este diseño
permite añadir un nuevo servicio a analizar con cierta independencia del
comportamiento de los agentes así como de la comunicación entre ellos.

Para la instrumentación de un servicio en el sistema es necesario establecer el
comportamiento de cada componente para realizar su función. Las particularidades
de cada componente así como los requerimientos para la puesta en práctica de un
nuevo servicio se explican a continuación.

Es importante destacar que para un funcionamiento consistente entre todas las
componentes del sistema, es necesario que tanto los servicios como los agentes
sean identificadas de un modo único dentro del mismo. Por esta razón cada
componente del sistema recibe como parámetro una llave que la identifica a ella
(\texttt{node\_key}) así como la llave del servicio que va a analizar
(\texttt{service\_key}).

\section{Collector}
El funcionamiento de la componente \textit{Collector} se definió en el módulo
\texttt{packages.collector}, que contiene una clase llamada \texttt{Collector}
que se encarga de realizar el procedimiento general de esta componente. O sea,
de llamar al método encargado de obtener una tupla cada cierto tiempo y una vez
obtenida esta, enviarla de forma distribuida a los \textit{storages} y los
\textit{analyzers} correspondientes a este en el sistema.  

Es importante destacar que la conformación de una tupla depende de las
caracteríticas del servicio en cuestión y el formato de las trazas que este
genera. Para la generalización de esta tarea se definió una \textit{interface}, 
llamada \texttt{DataCollector} compuesta únicamente por un método denominado
\texttt{collect\_data}. Este se encarga de la conformación de una tupla a partir
de los últimos registros recolectados en el archivo de trazas, desde la última
lectura o desde la hora de inicio de la componente. 

El tiempo que se debe esperar para conformar una nueva tupla es recibido como
parámetro (denominado \texttt{time\_delta}) en el constructor de la clase
\texttt{Collector}. La primera tupla se conforma con los datos recolectados
después de transcurrido dicho tiempo a partir del inicio de la ejecución de esta
componente. Luego, entre la conformación de una tupla y la siguiente se esperará
el tiempo estipulado. El valor asociado al parámetro \texttt{time\_delta} debe
ser un divisor del total de horas que conforman un día para garantizar que
exista la misma cantidad de intervalos cada día.

Las tuplas construidas por el \emph{DataCollector} deben contener la fecha y el
número del intervalo a los que corresponden el resto de los datos. Estos últimos
serán los valores asociados a las variables que sean seleccionadas para
caracterizar a dicho servicio. 

\section{Model Builder}
El funcionamiento del \textit{Model Builder} se definió en el módulo \\
\texttt{packages.model\_builder}, donde se declaró una clase llamada \\
\texttt{ModelBuilder} que se encarga de reconstruir cada cierto tiempo los
modelos de comportamientos ``normales'' de los servicios que le han sido
asociados. Para ello, obtiene los datos almacenados en los \textit{storages}
para cada uno de estos servicios y llama pasándole cada uno de estos conjuntos,
al método encargado de construir un modelo que represente los patrones de
comportamiento ``normal'' de un conjunto de datos. Luego, actualiza los modelos
asociados a cada servicio con los obtenidos para cada uno de ellos.

La construcción de un modelo que represente los patrones de comportamiento
``normal'' asociados a un conjunto de datos, dependiendo de la naturaleza de
estos, se puede realizar utilizando diferentes algoritmos. En conjunto, estos
pueden determinar la representación del modelo. Para permitir la utilización de
cualquier algoritmo y cualquier representación posible de un modelo, se definió
una \textit{interface} llamada \texttt{DataModelBuilder} compuesta únicamente
por un método denominado \texttt{get\_model}. Este se encarga de construir un
modelo de patrones de comportamiento ``normal'' para el conjunto de datos
recibido. 

El tiempo que se debe esperar para la reconstrucción de los modelos es recibido
como parámetro (se denominó \texttt{time}) en el constructor de la clase
\texttt{ModelBuilder}. La primera reconstrucción se realizará después de
transcurrido dicho tiempo a partir de la hora de inicio de esta componente.
Posteriormente, se esperara el tiempo estipulado entre una reconstrucción y
otra.

\section{Analyzer}
El funcionamiento de la componente \textit{Analyzer} se definió en el módulo
\texttt{packages.analyzer}, que contiene una clase llamada \texttt{Analyzer} que
se encarga de realizar el procedimiento general de esta componente. O sea,
obtener el modelo asociado al servicio al que pertenece la tupla que se está
analizando, llamar al método que se encarga de determinar si esta constituye o
no una anomalía basándose en dicho modelo, y posteriormente cambiar el tipo
asociado a esta por el obtenido por este método en los \textit{storages}. En
caso de que sea catalogada de anomalía esta será enviada de forma distribuida a
los \textit{classifiers} del sistema. 

La determinación de si una tupla constituye o no una anomalía depende de las
características del servicio y de la representación del modelo. Por este motivo,
para la generalización de esta tarea se definió una \textit{interface} llamada
\texttt{DataAnalyzer} compuesta únicamente por un método denominado
\texttt{analyze}. Este se encarga de determinar si los valores asociados a cada
una de las variables que caracterizan al servicio en dicha tupla se encuentran
dentro de los perfiles de normalidad. En caso de que todos se salgan de dichos
perfiles, se clasifica como anomalía y en cualquier otro caso se clasifican como
comportamiento ``normal''. 

\section{Classifier}
El funcionamiento de la componente \textit{Classifier} se definión en el módulo
\texttt{packages.classifier}, compuesto por una clase llamada
\texttt{Classifier} que realiza la llama al método encargado de clasificar las
anomalías detectadas y cambiar las clasificaciones asociadas a dicha tupla en
los \textit{storages} donde fue almacenada. 

La clasificación de una tupla depende de las características del servicio en
cuestión, por ello para garantizar la integración de cualquier implementación
posible para la realización de esta tarea se definió una \textit{interface}
denominada \texttt{DataClassifier}. Dicha interfaz está compuesta solamente por
un método llamado \texttt{classify} que se encarga de determinar las
clasificaciones asociadas a una anomalía teniendo en cuenta sus características.
Estas clasificaciones expresan las causas posibles o los errores subyacentes que
pudieron haber provocado su ocurrencia.


\section{Manager}
El funcionamiento de la componente \textit{Manager} se definió en el módulo
\texttt{packages.manager}, que contiene una clase llamada \texttt{Manager} que
se encarga del funcionamiento general de dicha componente. Esta realiza la
llamada al método que se encarga de determinar las acciones que se deben
realizar para recuperarse de la ocurrencia de una anomalía. Posteriormente,
cambia las acciones asociadas a dicha tupla en los \textit{storages} en que ha
sido almacenada y ejecuta las funciones asociadas a las mismas pasándole como
parámetro la tupla y las clasificaciones asociadas a la misma. Dichas funciones
deben encontrarse definidas en un módulo denominda \texttt{functions} dentro del
propio módulo del servicio en cuestión. 

Para la determinación de las acciones asociadas a una anomalía pueden utilizarse
varias metodologías y algortimos, por este motivo para su generalización se
definió una \textit{interface} denominada \texttt{DataManager} compuesta
únicamente por un método llamado \texttt{manage}. Este se encarga de determinar
las acciones asociadas a una anomalía teniendo en cuenta sus características y
las clasificaciones asociadas a esta. 

\section{Storage}
El funcionamiento de la componente \textit{Storage} se definió en el módulo
\texttt{packages.storage}, que contiene una clase llamada \texttt{Storage} que
se encarga de manejar las peticiones a la clase encargada del almacenamiento de
datos de los servicios que le han sido asociados en el sistema. 

Como se pudo observar en el capítulo anterior, la componente \textit{Storage} 
debe almacenar todas las tuplas recolectadas de los servicios que le han sido 
asociados, el tipo asociado a estas, las clasificaciones que le han sido 
asociadas a cada una de ellas, las acciones asociadas a cada clasificación y el 
modelo correspondiente a cada servicio. La forma de almacenar estas depende de 
las necesidades del usuario y de las características de los servicios. Por este 
motivo, para la generalización de estas tareas se definió una \textit{interface} 
llamada \texttt{DataStorage} compuesta por los siguientes métodos: 
\begin{itemize}
 \item \texttt{store(service\_key, tuple)}: Almacena la tupla dada asociada al
  servicio que representa la llave recibida. Es el método llamado por el
  \texttt{Collector} para almacenar una nueva tupla.
 \item \texttt{get\_data(service\_key)}:  Devuelve todas las tuplas asociadas al
  servicio que representa la llave dada. Es el método llamado por el
  \texttt{ModelBuilder} para la obtención de todos los datos asociados a un
  servicio específico.
 \item \texttt{get\_model(service\_key)}: Devuelve el modelo asociado al
  servicio que representa la llave dada. Este método es el que llama el
  \texttt{Analyzer} para obtener el modelo asociado a un servicio.
 \item \texttt{update\_model(service\_key, model)}: Actualiza el modelo asociado
  al servicio que representa la llave dada por el modelo recibido. Este método es
  el que llama el \texttt{ModelBuilder} para cambiar el modelo asociado a un
  servicio por el nuevo modelo construido.
 \item \texttt{set\_type(service\_key, tuple, type)}: Cambia el tipo asociado a
  a la tupla dada del servicio especificado por el tipo recibido. Este método es
  el que llama el \texttt{Analyzer} para cambiar el tipo asociado a una tupla.
 \item \texttt{get\_type(service\_key, tuple)}: Devuelve el tipo asociado a la
  tupla recibida del servicio dado. 
 \item \texttt{set\_classifications(service\_key, tuple, classifications)}:
  Cambia las clasificaciones asociadas a la tupla dada del servicio especificado
  por las las recibidas. Este método es el que llama el \texttt{Classifier} para
  cambiar las clasificaciones asociadas una tupla.
 \item \texttt{add\_classification(service\_key, tuple, classification)}: Añade 
  la clasificación recibida a la tupla dada del servicio especificado.
 \item \texttt{get\_classifications(service\_key, tuple)}: Devuelva las 
  clasificaciones asociadas a la tupla recibida del servicio dado.
 \item \texttt{set\_actions(service\_key, classification, actions)}: Cambia las 
  acciones asociadas a la clasificación dada del servicio especificado por las 
  recibidas. Este método es el que llama el \texttt{Manager} para cambiar las 
  acciones asociadas a una clasificación.
 \item \texttt{add\_action(service\_key, classification, action)}: Añade la
  acción recibida a la clasificación dada del servicio especificado.
 \item \texttt{get\_actions(service\_key, classification)}: Devuelve las
  acciones asociadas a la clasificación dada del servicio especificado. 
\end{itemize}