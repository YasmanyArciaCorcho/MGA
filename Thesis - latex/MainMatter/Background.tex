\chapter{Antecedentes}
\label{chap:background_project}
  En la Universidad de La Habana se brindan un conjunto de servicios de suma
importancia para el desarrollo científico-investigativo de los miembros de dicha
institución. Por este motivo, es necesario que sean gestionados, de manera que
las prestaciones que brinden estén acorde a las necesidades de su uso. 

Actualmente en la UH se cuenta con un conjunto de herramientas de gestión que le
permiten a los administradores de la red, realizar una detección temprana de
algunas situaciones anómalas en el comportamiento de la misma. Algunas de estas
herramientas ofrecen también un diagnóstico sobre la posible causa de la
anomalía, lo que posibilita que el problema sea solucionado rápidamente. Debido
a la importancia que tiene el proceso de gestión de los servicios brindados en
una red, así como el comportamiento en general de la misma, en la UH se decidió
incursionar en una nueva metodología para la detección de anomalías. 

La metodología diseñada\cite{Gonzalez2008} propuso un modelo capaz de detectar
anomalías a partir del uso de minería de datos en las trazas de los servicios.
Estas últimas ofrecen información sobre el comportamiento de estos, siendo
posible representar su estado en un intervalo de tiempo mediante la información
asociada a dicho período. Este modelo establece lo que es ``normal'' a lo largo
del tiempo, desarrollando y actualizando conjuntos de patrones con los que se
pueden comparar los eventos que se producen en los sistemas. Si uno de esos
eventos se sale del conjunto de normalidad, automáticamente se cataloga de
sospechoso. Para crear dichos conjuntos de patrones se utilizaron técnicas de
detección de \textit{outliers} para limpiar los datos, eliminando todos aquellos
valores que sólo aportaran ruido. 

El proceso anteriormente descrito, constituía una aplicación de minería de
datos, por ello para su desarrollo se planteó el seguimiento del modelo CRISP-DM
(\textit{CRoss Industry Standard Process for Data Mining})\cite{Wirth2000,
Hipp2002, Cios2005, Cios2010}. Dicho modelo, como se puede observar en la figura
\ref{fig:datamining}, ofrece una visión general del ciclo de vida de un proyecto
de minería de datos y está compuesto por las siguientes seis etapas:
\begin{enumerate}
 \item Definición del problema
 \item Definición de los datos
 \item Fase de preprocesamiento
 \item Creación del modelo
 \item Evaluación del modelo
 \item Aplicación o puesta en práctica
\end{enumerate}

\begin{figure}[!h]
 \begin{center} 
  \includegraphics[scale=.4, keepaspectratio=true]{Graphics/datamining} 
 \end{center}
 \caption{Fases del modelo CRISP-DM}
 \label{fig:datamining}
\end{figure}

\section{Definición del problema}
Para la validación de la metodología propuesta\cite{Gonzalez2008} se tomó como
caso de estudio el servicio de \textit{proxy}
Squid\footnote{\url{http://www.squid-cache.org/}}, utilizado en la UH como
intermediario para la navegación por Internet. Este servicio fue seleccionado
por ser el principal consumidor del ancho de banda de la red universitaria,
razón por la cual resulta importante contar con una herramienta capaz de
informar sobre cualquier problema en su funcionamiento. 

Como se planteó anteriormente, la metodología propuesta pretende detectar
anomalías a partir de las trazas de ejecución de los servicios. Para la gestión
del servicio de \textit{proxy} Squid es necesario obtener información útil para
la caracterización de dicho servicio a partir de las trazas que él genera. Para
ello fue necesario determinar que archivo de trazas sería utilizado, pues este
servicio genera tres archivos diferentes: \texttt{cache.log}, \texttt{store.log}
y \texttt{access.log}. En el primero se recopilan datos acerca de las
operaciones que la aplicación va realizando en su ejecución, \texttt{store.log}
contiene información acerca de los objetos que entran o abandonan la
\textit{cache} y el último, permite conocer información acerca de cada solicitud
realizada al servicio. Debido a los objetivos que se perseguían en dicho
trabajo, se seleccionó el archivo de trazas \texttt{access.log}.

Para la extracción de información útil para la gestión de dicho servicio, era
necesario conocer primero que datos contenía el fichero de trazas
\texttt{access.log} que se genera en la UH y el formato en que es almacenado
cada elemento. Con este objetivo se realizó un estudio del formato nativo de
Squid, utilizado en la UH, el cual contiene los siguientes campos:

\begin{center}
\scriptsize\ttfamily time duration client action/code bytes method url rfc931 
peerstatus/peerhost type
\end{center}

La descripción de estos campos sigue a continuación\cite{Wessels2004}: 
\begin{enumerate}
 \item \verb+time+: Es la fecha en la que el servidor está en condiciones de
  darle una respuesta al cliente sobre el pedido asociado a esta entrada. Este
  tiempo es expresado como la cantidad de milisegundos transcurridos desde el 
  1ro. de enero de 1970.
 \item \verb+duration+: Es el tiempo transcurrido desde que el cliente realiza
  la solicitud hasta el momento en que el servidor devuelve el recurso pedido.
 \item \verb+client address+: Es la dirección IP (\textit{Internet Protocol})
  del equipo que el cliente utiliza para realizar la solicitud.
 \item \verb+action+: Describe cómo fue tratada la solicitud localmente. Por
  ejemplo, si este campo tiene como valor \texttt{HIT}, significa que el recurso
  que se solicitó fue encontrado en el \textit{cache} y por lo tanto no fue
  necesario pedirlo a Internet. Asimismo, el valor \texttt{MISS} quiere decir 
  que el recurso no fue encontrado en \textit{cache}. Otros posibles valores se 
  pueden consultar en la página Web de las preguntas más frecuentes del servidor
  Squid~\footnote{\url{http://wiki.squid-cache.org/SquidFaq}}.
 \item \verb+code+: Es el código de respuesta HTTP (\textit{HiperText Transfer
  Protocol}) correspondiente a esta solicitud. En la especificación del estándar
  HTTP~\cite{Fielding1999} se definen los posibles valores para este campo.
 \item \verb+bytes+: Es la cantidad de datos en bytes de la respuesta entregada
  al cliente.
 \item \verb+request method+: Expresa el método HTTP que fue empleado para
  obtener la respuesta a la solicitud. Los valores más frecuentes son 
  \texttt{GET} y \texttt{POST}. En el estándar del protocolo HTTP se puede 
  encontrar el resto de los valores para este campo.~\cite{Fielding1999}.
 \item \verb+url+: Se refiere a la URL (\textit{Uniform Resource Locator})
  solicitada.
 \item \verb+rfc931+: Contiene información acerca de la identidad del cliente 
  si la solicitud fue autenticada. En caso contrario contiene el símbolo 
  \texttt{-}.
 \item \verb+peerstatus+: Con este campo se describe cómo y dónde fue 
  encontrada la respuesta a la solicitud.
 \item \verb+peerhost+: Es el nombre de la máquina donde fue encontrado el
  recurso aunque esto puede variar dependiendo de la versión de Squid que se 
  esté utilizando.
 \item \verb+content type+: Es el tipo de contenido del objeto solicitado, tal 
  y como aparece en la cabecera HTTP de la respuesta.
\end{enumerate}

Una vez conocidos los datos ofrecidos en el fichero de trazas
\texttt{access.log}, con el objetivo de enriquecer la información que se puede
extraer de estas para la gestión de este servicio; se formuló el siguiente
problema de minería de datos:

  Analizar las trazas de acceso del servicio \textit{proxy} Squid con el
  objetivo de:
  \begin{itemize}
    \item Identificar variables con un comportamiento estable en el tiempo. Es
     decir que fluctúen dentro de determinados parámetros acotables para 
     intervalos específicos de tiempo.
    \item Elaborar para cada una de las variables un perfil de comportamiento
     sobre la base del análisis de las mismas dentro de un conjunto inicial de
     archivos de trazas.
    \item Como vida útil de este conocimiento extraído, examinar nuevos
     registros de Squid con el objetivo de buscar un valor para dichas variables
     dentro de estos nuevos ficheros y compararlos con los perfiles definidos 
     para cada una de ellas.
  \end{itemize}

\section{Definición de los datos}
\label{sec:data_definition}
Para la aplicación de esta fase al problema que se estaba resolviendo, se
identificó en una primera instancia un conjunto de variables o atributos a
partir de toda la información que se brinda en las trazas de acceso de Squid.
Para su elección no se tuvo en cuenta ninguna característica especial, aunque 
se consideró el hecho de que estas se encontraban muy relacionadas con el
comportamiento del ancho de banda de Internet que tiene la universidad. Estas
variables fueron:
\begin{itemize}
 \item cantidad total de solicitudes realizadas (\texttt{cs})
 \item cantidad total de solicitudes que no se encontraron en \textit{cache} 
  (\texttt{css})
 \item cantidad total de bytes entregados a los clientes (\texttt{cb})
 \item cantidad total de bytes bajados directamente de Internet (\texttt{cbs})
\end{itemize}

Para construir la colección de datos inicial se analizaron las trazas de acceso
en un intervalo de $150$ días de ejecución del servicio Squid y se agruparon los
datos en intervalos de $5$ minutos; obteniéndose como resultado un número fijo
de entradas diarias\footnote{Es decir, que a diferencia de los datos iniciales,
ahora cada día consta de una cantidad fija de registros (288 en este caso)}. Por
cada uno de estos intervalos se contabilizaron resúmenes para las cuatro
variables seleccionadas. Esta información se almacenó en tablas separadas (una
por cada variable) dentro de una base de datos relacional. Cada una de estas
tablas contiene solamente tres campos. Estos son: la fecha, el número del
intervalo y el valor de la variable. Por ejemplo, la tupla \verb+(20120325, 1, 75)+, 
dentro de la tabla \texttt{cs}, significa que el día 25 de Marzo de 2012,
entre las 12:00 y 12:05 am (equivalente al primer intervalo del día) se
realizaron $75$ solicitudes. Nótese, que cada tupla tiene asociado el número del
intervalo del día al que corresponden sus datos.

\section{Fase de preprocesamiento}
A partir de la colección de datos construida en la fase anterior se graficó el
comportamiento de las variables por intervalos para todos los días de la muestra
analizada. En estas gráficas, se pudo observar que aunque la mayoría de los
puntos seguían un mismo patrón, varios puntos se alejaban del resto. Observese
el comportamiento de la variable \texttt{cs} en la figura \ref{fig:cs_monday}
durante 21 lunes; como se puede apreciar existen puntos que se alejan del patrón
común. Estos puntos, pueden ser vistos como anomalías u \textit{outliers}, y no
deben tenerse en cuenta en la construcción del modelo. Para su identificación se
utilizaron algunas técnicas de detección de \textit{outliers}, obteniéndose
buenos resultados con el algoritmo \texttt{DBSCAN} (\textit{Density Based
Spatial Clustering of Applications with Noise })\cite{Ester1996}. Luego, se
conformó el conjunto de datos final con todos los elementos que no fueron
clasificados como ruido u \textit{outlier}, para la posterior construcción del
modelo a partir de este. %TODO: Cambiar figura monday.

\begin{figure}[!h]
 \begin{center}
  \includegraphics[scale=.5, keepaspectratio=true]{Graphics/monday}  
 \end{center}
 \caption{Cantidad de solicitudes durante 21 lunes.}
 \label{fig:cs_monday}
\end{figure}

\section{Creación del modelo}
En la realización de esta fase se trabaja con el conjunto de datos final
obtenido a partir del algoritmo \texttt{DBSCAN}. Finalmente, el modelo que
representaba el comportamiento normal de un servicio estaba dado por umbrales
(mínimo, máximo) definidos para cada variable en cada intervalo de un día de la
semana. Esta decisión se tomó basándose en las características de los datos y
del problema en sí mismo, pues justamente se querían definir ciertas cotas por
cada intervalo que permitieran establecer la diferencia entre un comportamiento
normal y una posible anomalía en el servicio.

Para almacenar los perfiles se usaron diferentes tablas (una por cada variable) 
dentro de una base de datos relacional. Las columnas de estas eran:
\begin{itemize}
 \item \texttt{weekday}: día de la semana (lunes = 0, martes = 1, \ldots, 
  domingo = 6)
 \item \texttt{interval\_number}: número del intervalo dentro del día de la semana 
  que se está analizando.
 \item \texttt{min}: el mínimo valor de los datos
 \item \texttt{max}: el máximo valor de los datos
\end{itemize}

\section{Evaluación del modelo}
Después de creado el modelo surgió la incógnita de si entre las variables
seleccionadas -dígase \texttt{cs}, \texttt{css}, \texttt{cb} y \texttt{cbs}-
existía solapamiento de información y por ende se podían establecer conclusiones
entre ellas. Para ello, se realizó un estudio sobre dichas variables, en el cual
se graficó la interrelación de las mismas en un período de tiempo y se observó
que entre cada pareja de variables parecía existir una correlación lineal. Por
ese motivo se calculó el coeficiente de correlación lineal entre estas para
comprobar dicha hipótesis. Los resultados se pueden ver en la tabla
\ref{tab:correlation}. Después de observados los valores de la tabla, se puede
decir, bajo el criterio de la existencia o no de correlación lineal a partir de
un valor de $r$, que todas las posibles combinaciones con las variables parecen
tener una relación lineal positiva. Sin embargo, usualmente se considera que
existe correlación lineal cuando $r > 0.75$\cite{Cue1989}. Por tanto, bajo este
criterio solamente se consideraron correlacionadas las parejas $(cs, css)$ y
$(cb, cbs)$. %TODO: Arreglar referencia Cue1989.

\begin{table}[ht]
 \begin{center}
  \begin{tabular}{|c|c|}
   \hline
    Variables
   & $r$
   \\\hline
   $(cs, \ \ cb)$
   & 6.077598e-01 
   \\\hline
   $( cs, \ css)$  
   & 9.922090e-01
   \\\hline
   $(cs,\  cbs)$
   & 5.172269e-01
   \\\hline
   $(cb,\ css)$
   & 6.052734e-01
   \\\hline
   $(cb,\  cbs)$
   & 9.880280e-01
   \\\hline
   $( css,  cbs)$
   & 5.181871e-01
   \\\hline
  \end{tabular}
 \end{center}
 \caption{Cálculo del coeficiente de correlación}
 \label{tab:correlation}
\end{table}

En consecuencia se pensó en comprobar sólo si una de las variables de las
parejas correlacionadas se ajustaba al perfil asociado a ella, para catalogar de
inusual el comportamiento de un servicio en un período de tiempo. Pero en el
problema que se está tratando de resolver sólo se desea informar a los
administradores de la red cuando se tiene una fuerte sospecha de que ha ocurrido
una anomalía. En este sentido se decidió clasificar como anomalía a aquellas
tuplas $(cs, css, cb, cbs)$ donde las cuatro variables queden fuera del perfil
construido para dicho intervalo de tiempo.

\section{Aplicación o puesta en práctica}
Para esta fase se desarrolló una aplicación\cite{Gonzalez2008} que permitiera
validar la metodología propuesta. Esta aplicación tan sólo detecta las anomalías
correspondientes al servicio \textit{proxy} Squid de la UH, generando reportes
sobre las anomalías detectadas en el comportamiento del mismo. La misma obtiene
las trazas del servidor Squid y va formando tuplas con la información acumulada
por intervalos de $5$ minutos. Luego, estas tuplas son clasificadas en anomalías
o comportamiento ``normal'' sobre la base de conocer si se ajustan o no al
perfil modelado. 

Como se mencionó anteriormente esta aplicación debía realizar dos tareas
principales; la primera era obtener los valores asociados a las variables
definidas en la sección \ref{sec:data_definition}, a partir de los datos de las
trazas de ejecución. La segunda era determinar si un nuevo evento constituye o
no una anomalía a partir del modelo asociado al comportamiento ``normal'' de
este servicio. De ahí que fueran identificadas dos componentes principales en
esta aplicación: un \emph{Recolector de datos} y un \emph{Clasificador}. La
comunicación entre estas componentes se realizó siguiendo el modelo
cliente-servidor. Finalmente, la arquitectura adoptada quedó como se ilustra en
la figura \ref{fig:architecture}.

\begin{figure}[ht]
 \begin{center}
  \includegraphics[scale=.6, keepaspectratio=true]{Graphics/architecture}
 \end{center}
 \caption{Arquitectura de la aplicación.}
 \label{fig:architecture}
\end{figure}

El \emph{Recolector de datos} es el encargado de leer los registros que van
apareciendo en el archivo de trazas de dicho servicio, conformar las tuplas que
caracterizarán el comportamiento del mismo durante un cierto intervalo de tiempo
y enviarlas al \emph{Clasificador}. Para los objetivos de esta aplicación una
tupla era una lista de valores con la siguiente información:
$$ (\texttt{date, interval\_number, cs\_value, css\_value, cb\_value, cbs\_value})$$

donde \texttt{date} es la fecha de esa tupla, \texttt{interval\_number} se
refiere al intervalo de esa fecha que se está analizando, y los cuatro campos
restantes son los valores asociados a las variables que representarán a dicho
servicio.

Con cada nueva ejecución de este módulo, se lee cada línea que el servicio de
Squid va escribiendo al final del fichero de trazas \texttt{access.log} y se
averigua la hora de dicha entrada. A partir de este valor se obtiene  la fecha
de comienzo del intervalo más cercano y se empiezan a leer los registros de las
trazas agrupándolos por intervalos de $5$ minutos. Una vez que se tiene
almacenado el grupo de registros del intervalo, se averigua la fecha, y los
demás valores de la tupla se obtienen contabilizando resúmenes a partir del
grupo de registros. Es importante destacar que este módulo debe ejecutarse en el
mismo servidor donde esté funcionando el servicio de Squid, para evitar el
retraso en las lecturas por la transferencia a través de la red.

Para instrumentar una lectura continua del archivo de trazas, en la medida en
que este iba creciendo, se implementó una funcionalidad similar a la del comando
\texttt{tail}\footnote{\texttt{tail} muestra las últimas líneas de un fichero y
si se le especifica la opción \texttt{- f} muestra las líneas en la medida en
que va creciendo} de Linux. 

Una vez obtenida una tupla de valores listos para clasificar se pasa la
información utilizando XML-RPC\footnote{XML-RPC es un protocolo de llamada a
procedimiento remoto que usa XML para codificar los datos y HTTP como protocolo
de transmisión de mensajes. Con él, un cliente puede invocar métodos con
parámtetros en un servidor remoto y obtener información estructurada como
respuesta.} al módulo \emph{Clasificador}, el cual debe determinar si esta
constituye o no una anomalía. Para ello deben obtenerse los perfiles
establecidos por el modelo construido, para cada variable y verificarse si esta
tupla se adecua a los patrones de comportamiento ``normal'' establecidos para
este servicio. Una vez clasificada se almacena esta información en una base de
datos, para permitir la retroalimentación del modelo en una fase posterior.
Aunque en esta implementación no se contempló esta tarea dentro de la
aplicación.

La aplicación propuesta constituye un sistema centralizado, donde si alguna de
estas componentes deja de funcionar, el sistema completo también lo hará. En
caso de que el servicio que se está analizando estuviese corriendo en más de un
servidor, los \emph{recolectores de datos} que se encuentran en estos enviarían
sus datos al mismo \emph{Clasificador}. Luego, se crearía una sobrecarga en esta
componente, haciendo su funcionamiento más lento y creando en el
\emph{Clasificador} del sistema lo que se conoce como ``cuello de
botella''\footnote{TODO: Significado de cuello de botella}.

\section{Conclusiones}
La aplicación desarrollada brindó un sistema capaz de detectar anomalías en el
comportamiento del servicio \textit{proxy} Squid, de la UH, validando así la
metodología propuesta. Dicha herramienta fue creada con un motivo puramente
experimental por lo que en su diseño no se concibieron algunos aspectos vitales
para su uso diario en esta institución. 

Una de estas deficiencias radica en que el modelo no se retroalimenta de los
cambios que se producen en el comportamiento de este servcio. En consecuencia,
si el modelo inicial representa el comportamiento de este servicio durante los
meses de clases cuando se realizan muchas solicitudes, en los meses de
vacaciones se detectarán muchas anomalías debido a la disminución lógica del
volumen de peticiones. De igual manera, también serán ignorados los cambios que
se produzcan de un año a otro en su comportamiento. Ambos hechos producen no
sólo la clasificación en anomalías de elementos que no se salen del
comportamiento esperado, sino también la no detección de eventos que si se
alejan del perfil de normalidad esperado para el mismo. 

Por otro lado, esta aplicación no brinda la posibilidad de incorporar otro
servicio para ser gestionado por la misma. Aunque sí mostró una metodología
acertada para el diseño y la instrumentación de una herramienta para la
detección de anomalías en cualquier servicio de la red. Por este motivo, tomando
como base las ideas planteadas por la metodología propuesta y la aplicación
resultante; se pretende implementar un sistema para la detección de anomalías en
los servicios de la red. 

Debido a los problemas que presenta una arquitectura centralizada para este tipo
de sistemas, se decidió diseñar un sistema distribuido que permita aprovechar
los recursos disponibles en las computadoras que lo integran en pos de su
funcionamiento. Este paradigma posibilita además la distribución de sus tareas,
de forma que se balancée la carga del mismo; incrementándose así su eficacia y
eficiencia.

El sistema que se pretende implementar comprenderá las fases de la metodología
planteada anteriormente y la retroalimentación del modelo, con el objetivo de
posibilitar la adaptación a los cambios que se produzcan en el comportamiento de
los servicios a lo largo del tiempo. Además, para facilitarle el trabajo a los
administradores se pretende que dicho sistema sea capaz de determinar de forma
semiautomática o automática las acciones a realizar para recuperarse de la
ocurrencia de una anomalía. Para ello, se estipuló la clasificación de los
comportamientos inusuales, de forma que se puedan determinar las acciones a
realizar para la recuperación ante su ocurrencia. Dicha herramienta notificará
inmediatamente de la detección de una anomalía y brindará la posibilidad de
darle seguimiento a cualquier servicio siempre que se brinden los elementos
necesarios para su funcionamiento. En capítulos posteriores se explicará el
diseño de dicho sistema, así como los elementos necesarios para la
instrumentación de la gestión de un nuevo servicio como parte del mismo.
%%% Local Variables:
%%% mode: latex
%%% TeX-master: "../Thesis"
%%% End:
