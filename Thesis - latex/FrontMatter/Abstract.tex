\begin{abstract}
La implementación de un Lenguaje de Dominio Especifico (DSL) requiere diseñar un programa capaz de leer líneas de texto escritas en dicho DSL, analizarlas, procesarlas y hacer posible su generación de código o implementación hacia otros lenguajes. En dependencia de su propósito, desarrollar un DSL puede necesitar varias etapas, pero la mayoría de estas fases son comunes en todas las implementaciones. En este trabajo se presenta una herramienta para automatizar la creación de DSLs que generen códigos hacia múltiples lenguajes. Dicha herramienta permite a los desarrolladores abstraerse de la implementación de analizadores lexicográficos y semánticos, automatizar la construcción de los nodos del Árbol de Sintaxis Abstracta del lenguaje a diseñar y simplificar la generación de código a los lenguajes que se desee exportar.
\end{abstract}


\begin{englishabstract}
The implementation of a Specific Domain Language (DSL) requires designing a program capable to read lines
of text written in that DSL, analyze and process them and make possible its code generation or implementation
into other languages. Depending on the purpose of the DSL, developing it may need several stages, but most of
these stages are common in all implementations. This paper presents a tool to automate the creation of DSLs
that generate codes into multiple languages. This tool allows developers to abstain from the implementation
of lexical and semantic analyzers, automate the construction of the Abstract Syntax Tree nodes of the
language to be designed and simplify the code generation of the languages to be exported.	
\end{englishabstract}