\chapter{GAGM Avanzada}
\label{chap:GAGMA}

\agregaesto{PON ESTO EN OTRO FICHERO}

Los macros \texttt{defnode} y \texttt{gcode} utilizan la información almacenada en los parámetros de los llamados a dichos macros, para definir las clases, funciones constructoras o realizar la generación de código. Por defecto, si no se especifican las opciones de los slots, las palabras claves, el Lambda List y el cuerpo de las funciones constructoras y de los métodos; esta información será rellenada automáticamente. Debido a que los usuarios pueden necesitar distintas funcionalidades en su dominio, en la próxima sección se presentan otras funcionalidades del macro \texttt{defnode} que permiten modificar el código generado por él. \agregaesto{SI TAMBIÉN VAS A HABLAR DE GCODE, DILO AQUÍ.}

\section{Opciones de DefNode}
\label{defnodeavanzado}
En la definición de las clases y funciones constructoras existen un conjunto de propiedades que Defnode completa con valores prefijados. A continuación, se presentará una opción para definir los slots de las clases y posteriormente las opciones que permiten modificar el código generado por DefNode. 


\agregaesto{NO EMPIECES CON EL LAMBDA LIST DEL DEFNODE NI DE NADA. EXPLICA CON PALABRAS LO QUE QUIERES HACER.}. Lambda List del Defnode:  
\begin{verbatim}
   (Nombre-de-la-clase (lista-de-herencia) (definición-de-slots) 
                       (parámetros-opcionales))
\end{verbatim}

\agregaesto{EL NOMBRE DE LA CLASE Y LA LISTA DE HERENCIA SON IGUALES A LO QUE SE HACE CON EL DEFCLASS, pero en la definición de slots y parámetros opcionales hay cambios. en las siguientes secciones se muestran estos cambios.}

\agregaesto{Vaselina a lo que sigue.}

\subsection{Slots}
Cada elemento de la lista definición-de-slots hace referencia a la declaración de un slot en la definición de la clase Nombre-de-la-clase, y puede ser un símbolo o una lista. Si dicho elemento es un símbolo, la macro expansión de DefNode rellena las opciones ACCESSOR e INITARG con el mismo nombre del slot; si es una lista, se deben definir las opciones de slots que se deseen y sus respectivos valores.  

En el siguiente ejemplo se muestra la definición de una clase con tres slots. 
\begin{verbatim}
        (DefNode collection-definition-node () 
	  	      (collection-name
               collection-type 
	  	       (collection-capacity 
	  	                          :initarg :capacity 
	  	                          :initform 100))
              ())
\end{verbatim}

Los dos primeros serán rellenados con los valores por defecto, pero en el caso del tercero, como se definen dos propiedades el initarg y el initform, esas se agregan a su definición, como se puede apreciar en la macroexpansión del código anterior:

Donde un fragmento de su macro expansión es:
\begin{verbatim}
	     (defclass collection-definition-node () 
	          ((collection-name 
	                 :accessor  collection-name
	                 :initarg  :collection-name 
	                 :allocation instance 
	                 :initform nil 
	                 :documentation “emtpy”)
	           (collection-type 
	                 :accessor  collection-type
	                 :initarg  :collection-type 
	                 :allocation instance 
	                 :initform nil 
	                 :documentation “emtpy”)
	            (collection-capacity 
	                 :accessor collection-capacity
	                 :initarg: capacity
                     :allocation instance
	                 :initform 100
	                 :documentation “empty”))
\end{verbatim}

DefNode emplea un conjunto de palabras claves para agregar o modificar el código que genera su expansión. Ellas son especificadas en el último argumento del Lambda List de dicho macro a través de parámetros nombrados, los cuales se presentan a continuación.

\subsection{Nombre y Tipo de función constructora}
El nombre para la función constructora de una clase se especifica mediante la palabra clave func-name. Las funciones constructoras pueden ser macros, funciones o métodos. A través de la palabra clave ctr-type se especifica el tipo de constructor (defmacro, defun, defmethod).
Ejemplo:
\begin{verbatim}
	(DefNode collection-insert-node () (name value) 
	(:ctr-type funtion :func-name collection-insert))
\end{verbatim}

\subsection{Lambda List de la función constructora}
Por defecto, el Lambda List de las funciones constructoras está formado por los nombres de las opciones Initarg de los slots de la clase a instanciar. Ellos se encuentran ordenados como fueron declarados en su definición.\\ 
Ejemplo:
\begin{verbatim}
	(defnode collection-insert-node () 
	  (collection-name collection-name 
	       (element :initarg :element :initform nil))
	  (:func-name collection-insert)))
\end{verbatim}
Un fragmento de su macro expansión es:
\begin{verbatim}
(defun collection-insert (collection-name  element))
\end{verbatim}
La palabra clave lambda-list permite definir el Lambda List de la función constructora. En este caso, el generado por DefNode se sustituye por la nueva especificación de los parámetros.\\ 
Ejemplo:\\
Si se desea que la clase collection-insert-node reciba un número arbitrario de argumentos, se define su Lambda List como se muestra a continuación.
\begin{verbatim}
(defnode collection-insert-node () (reference-collection value) 
                                   (:ctr-type funtion 
                                    :func-name collection-insert-range 
                                    :lambda-list (&rest value)))
\end{verbatim}
Un fragmento de su macro expansión sería:
\begin{verbatim}
(defun collection-insert-range (&rest value))
\end{verbatim}

Y a partir de esta definición se puede llamar como: \agregaesto{PON UN EJEMPLO DE CÓMO SE LLAMARÍA A LA FUNCIÓN. SERÍA BUENO, UN EJEMPLO DE CÓMO SE PUEDE USAR EL CONSTRUCTOR POR DEFECTO, Y LOS CAMBIOS QUE LE PODEMOS HACER NOSOTROS.}

\subsection{Ctr-body}

El cuerpo de la función constructora se puede especificar con la palabra clave ctr-body.
Ejemplo:
\begin{verbatim}
(defnode collection-insert-node ()(reference-collection value)
								(:ctr-body (make-instance 'collection-insert-node
								               :reference-collection reference-collection)
								               :value value))
\end{verbatim}
En este caso, el valor de ctr-body puede ser cualquier fragmento de código Lisp válido.


%\agregaesto{FINAL DE SECCIÓN}

%%% Local Variables:
%%% mode: latex
%%% TeX-master: "../Thesis"
%%% End: